% Created 2016-05-17 Tue 21:27
\documentclass[11pt]{article}
\usepackage[utf8]{inputenc}
\usepackage[T1]{fontenc}
\usepackage{fixltx2e}
\usepackage{graphicx}
\usepackage{longtable}
\usepackage{float}
\usepackage{wrapfig}
\usepackage{rotating}
\usepackage[normalem]{ulem}
\usepackage{amsmath}
\usepackage{textcomp}
\usepackage{marvosym}
\usepackage{wasysym}
\usepackage{amssymb}
\usepackage{hyperref}
\tolerance=1000
\author{谢劲  \url{https://github.com/eric8180/Notes.git}}
\date{\today}
\title{NOTES}
\hypersetup{
  pdfkeywords={},
  pdfsubject={},
  pdfcreator={Emacs 24.5.1 (Org mode 8.2.10)}}
\begin{document}

\maketitle
\tableofcontents



\section{eric's note}
\label{sec-1}
\subsection{plan}
\label{sec-1-1}
\subsubsection{time management}
\label{sec-1-1-1}

\begin{itemize}
\item time mangagement
\begin{center}
\begin{tabular}{lll}
 & 工作时间 & 运动时间\\
\hline
工作日 & 8-10 hours & 0.5 hour\\
周末 & 2-6 hours & 0.5 hour\\
法定节假日 & 随意 & 随意\\
\end{tabular}
\end{center}
\end{itemize}



\begin{itemize}
\item checkout
\begin{center}
\begin{tabular}{llllllll}
日期 & 考核 &  & 日期 & 考核 &  & 日期 & 考核\\
2015/5/4 & B+ &  & 2015/5/10 & D &  & 2015/5/16 & B+\\
2016/5/5 & C &  & 2015/5/11 & D &  & 2015/5/17 & B\\
2016/5/6 & D &  & 2015/5/12 & D &  &  & \\
2016/5/7 & D &  & 2015/5/13 & C &  &  & \\
2016/5/8 & D &  & 2015/5/14 & D &  &  & \\
2016/5/9 & D &  & 2015/5/15 & C &  &  & \\
\end{tabular}
\end{center}
\end{itemize}

\subsubsection{reading}
\label{sec-1-1-2}

\begin{center}
\begin{tabular}{ll}
书名称 & 阅读计划\\
\\
\end{tabular}
\end{center}

\subsection{reserve}
\label{sec-1-2}
\subsubsection{concept}
\label{sec-1-2-1}
\begin{itemize}
\item EPOC (excess post - exercise oxygen consumption)
运动后过量氧耗:运动后恢复期内为了偿还运动中的氧亏,以及在运动后使处于高水平代
谢的机体恢复到安静水平时消耗的氧量,称为运动后过量氧耗。
\item UEFI,全称Unified Extensible Firmware Interface,即“统一的可扩展固件接口”,是一种详
细描述全新类型接口的标准,是适用于电脑的标准固件接口,旨在代替BIOS(基本输入/输出系
统)。此标准由UEFI联盟中的140多个技术公司共同创建,其中包括微软公司。UEFI旨在提高软
件互操作性和解决BIOS的局限性。
\item 物联网(概念):顾名思义,物联网就是物物相连的互联网。这有两层意思:其一,物联网的核心和基础仍然
\end{itemize}
是互联网,是在互联网基础上的延伸和扩展的网络;其二,其用户端延伸和扩展到了任何物品与物品
之间,进行信息交换和通信,也就是物物相息。物联网通过智能感知、识别技术与普适计算等通信感
知技术,广泛应用于网络的融合中,也因此被称为继计算机、互联网之后世界信息产业发展的第三次
浪潮。
实际应用:中国的第二代身份证,第二代身份证最显著的进步不是说在卡表面的照片换为彩色的了,而是在卡的内部更富科技含量的RFID芯片。芯片可以存储个人的基本信息,需要时在读写器上一扫,即可显示出你身份的基本信息。而且可以做到有效防伪,因为芯片的信息编写格式内容等只有特定厂家提供,伪造起来技术门槛比较高。
中国大部分高校的学生证,由于中国的高校生数量众多,学生假期返乡出行享受火车半价优惠。为此,相关部门采用了可读写的RFID芯片。里面存贮了该用户列车使用次数信息,每使用一次就减少一次。而且不易伪造,加强了管理。
很多的一卡通比如,市政一卡通、校园一卡通都可以归为较为简单的物联网应用。
发展:无人驾驶。
\subsubsection{linux}
\label{sec-1-2-2}
\begin{itemize}
\item Linux文件系统以及目录结构简介
\url{http://blog.chinaunix.net/uid-9525959-id-2001812.html}
其中Linux与Windows文件系统不同的是,Linux没有硬盘之分全是文件,但是每个文件里装的东西
有个大概的分类。比如/boot里面放的就是引导。
\item 设置新的环境变量:env指令是显示当前用户的环境变量。
gedit /etc/profile
增加 DEBIAN \_ BACKUP="/media/学习" (变量)
终端输入 source /etc/profile  (使变量即可生效)
echo \$DEBIAN \_ BACKUP 是否设置正确
重启后永久生效
\item fat文件系统采取链表的方式存放文件数据,读取所有数据需要依次遍历所有节点,因此当文件过散的
时候我们读取该文件的所有数据需要硬盘转好几圈,因此windows会经常需要磁盘碎片整理。ext文件
系统则不需要ext文件系统中,有inode 和block。其中每个文件(档案和目录)对应唯一一个inode,这个inode中保存
了文件所有的block。(block负责存放文件具体的数据)。不同的是对于档案block保存的是档案数
据,对于目录block保存的则是相应档案或子目录的inode
\item 退出vi的方法:esc + : + q!+enter
\item gnome-open /etc \#shell指令打开图形界面的/etc
\item 在正则表达式中 ’\^{} g’ 与‘[\^{} g]’ 其中的 \^{}具有不同的含义。单引号中的\^{} 表示行头,[]中的\^{} 表示取反
关于linux 用户和群组,在当前用户指令下输入groups可以知道当前用户加入了哪些组,当用户想加入特定的组是可以通过两种方式:
第一:指令 gpasswd -a userid groupid(推荐第一种,因为有的时候文本显示的东西在更改
后不会实际产生作用,特殊情况除外所以统一用指令来更改万无一失)
第二:在/etc/group 文件相应的group最后一列加入用户id 再保存
groups指令可能显示可能会出现错误,在加入或删除组名时没有及时更新
通过newgrp groupsid 可以更改当前用户的有效群组 /etc/passwd 文件存放了用户id的相关信息
/etc/group 文件存放了groupid 的相关信息
\item 非root用户想使用sudo 必须在 /etc/sudoers 加入"用户id ALL=(ALL:ALL) ALL"语句 其中ALL可
  以设置为某一条命令(绝对路径)这样就可以限制该用户利用sudo使用什么指令
linux中可以通过终端给不同的用户发消息,利用指令who 得到目前在线的用户和端口号,通过 指
令 write userid userport就可以开始给别的用户发送消息 ctrl+D结束消息输入
\item 总的来说,CPU从内存中一条一条地取出指令和相应的数据,按指令操作码的规定,对数据进行运算处理,直到程序执行完毕为止。CPU的运行原理就是:控制单元在时序脉冲的作用下,将指令计数器里所
指向的指令地址(这个地址是在内存里的)送到地址总线上去,然后CPU将这个地址里的指令读到指令寄存器进行译码。对于执行指令过程中所需要
用到的数据,会将数据地址也送到地址总线,然后CPU把数据读到CPU的内部存储单元(就是内部寄存器)暂存起来,最后命令运算单元对数据进行处
理加工。周而复始,一直这样执行下去,天荒地老,海枯枝烂,直到停电。来自 \url{http://blog.chinaunix.net/uid-23069658-id-3563960.html}
\end{itemize}
\subsubsection{emacs}
\label{sec-1-2-3}
\begin{enumerate}
\item org-mode
\label{sec-1-2-3-1}
\begin{itemize}
\item org-mode打开的时候只显示一级标题后面有.. 表示有内容。按 \textbf{TAB} 键可以打开或者关闭目录。按 \textbf{shift+tab} 打开全部目录
\item 将org文件导出为其他文件的方法:C-c C-e接着按选项选择
\item 列举内容时,假如要加序号,则序号一定要对齐并且列表后面要加 \textbf{空格} 不然无法正常显示出来。
\item 当文本内容中想输入“\_ ”(下标) " \^{} "(上标) 等表示字体的符号时,记得加空格
\item alt + ret 插入一个同级标题
\item 文档元数据包括TITLE,AUTHOR等。使用时用 \#+TITLE: 注意:要紧跟title变颜色后才成功。更多元数据见网址 \url{http://www.360doc.com/content/14/1219/13/20545288_434126794.shtml}
\item C-c C-t 改变当前条目状态(TODO DONE NULL) C-c C-d 增加截止期限 C-c C-s增加日程安排
\end{itemize}
\item other
\label{sec-1-2-3-2}
\begin{itemize}
\item melt+>可以移动到文本末尾 melt+<可以移动到文本开头
ctrl+a可以移动到本行开头 ctrl+e可以移动到本行末尾
\item C-x C-q 可以将缓冲区切换为只读缓冲区或者取消
\end{itemize}
\end{enumerate}

\subsubsection{python}
\label{sec-1-2-4}
\subsubsection{windows}
\label{sec-1-2-5}
\begin{itemize}
\item 如何在cmd中增加新命令:emacs命令。打开系统属性(在搜索框中搜索path)-》环境变量-》
\end{itemize}
选中path并点编辑-》增加emacs应用程序所在目录并以$\backslash$结尾。这样就可以在CMD中直接输入
emacs来启动了
\begin{itemize}
\item 删除右键git bash选项:打开注册表(cmd中输入regedit) 找到并删除\HKEY$_{\text{LOCAL}}$$_{\text{MACHINE}\SOFTWARE\Classes\Directory\backtory\git}$$_{\text{bash}}$
\end{itemize}
\subsubsection{SQL}
\label{sec-1-2-6}
\begin{itemize}
\item 关系数据库包括并,差,交,笛卡尔积,投影,除以及 \textbf{连接} 关系。在连接关系中通过主表的主键
与从表的外键建立连接。(外键必须是从表的主键或者唯一值)
\item 建立E-R模型时注意优化表格达到第三范式,主键确定则其它列的值也确定了我们称之为第二范式,第三范式则是消除了传递性依赖的第二范式
\item 分组计算:计算函数和GROUP BY 命令组合,当分组需要加判定条件时使用HAVING而不是WHERE
\item 
\end{itemize}

\subsubsection{GIT}
\label{sec-1-2-7}
\begin{itemize}
\item git clone时遇见error setting certificate verify locations错误时,可以尝试
\end{itemize}
用 git config --global http.sslVerify false 来解决
\begin{itemize}
\item git三部曲 git add(添加文件到缓冲区)->git commit(缓冲区文件到本地库)->
\end{itemize}
git push(上传改动到服务器)。其中可以通过git status查看状态
\begin{itemize}
\item git push 方法
\item 通过指令git remote add "分支名称" "仓库URL" // 添加push仓库对应的名称
\item 通过指令git push "分支名称" // push 文件到仓库(会提示输入仓库的用户名和密码)
\item 详情见\url{http://my.oschina.net/u/1050949/blog/194536} 
\begin{itemize}
\item git 不设置代理方法:git config --global --unset http.proxy
git config --global --unset https.proxy
\end{itemize}
\item git clone经常连接不上的解决方法:git config --global http.postBuffer 52488000
\item 
\end{itemize}




\subsubsection{C\#}
\label{sec-1-2-8}
\begin{itemize}
\item 
\end{itemize}

\subsection{other}
\label{sec-1-3}
\begin{itemize}
\item iphone icloud 连接不了服务器解决方案:\url{http://jingyan.baidu.com/article/f0e83a25a58dad22e59101dd.html}
                                       管理员身份运行CMD(右键选择)-》netsh winsock reset-》重启电脑
\end{itemize}




\section{Todo}
\label{sec-2}
\subsection{{\bfseries\sffamily DONE} Things with days}
\label{sec-2-1}
\begin{itemize}
\item State "DONE"       from "TODO"       \textit{[2016-05-17 Tue 17:57]}
\begin{itemize}
\item 更新emacs,并使用陈斌的配置文件
\item 
\end{itemize}
\end{itemize}
\subsection{{\bfseries\sffamily CANCELLED} Things with a week}
\label{sec-2-2}
\begin{itemize}
\item State "DONE"       from ""           [2016-05-17 Tue 17:
\end{itemize}
% Emacs 24.5.1 (Org mode 8.2.10)
\end{document}